\chapter*{\RL{ملخص}}
\addcontentsline{toc}{chapter}{Arabic Abstract}

\begin{RLtext}
يعكس هذا التقرير العمل الذي قمت به في أورانج بيزنس المغرب في إطار من مشروعي الختامي للحصول على شهادة مهندس دولة في الاتصالات وتكنولوجيا المعلومات.

 في سياق تهيمن فيه الهياكل الموزعة وممارسات التطوير السريعة على تصميم التطبيقات، أصبح دمج مبادئ تكنولوجيا الخضراء في ممارسات \LR{DevOps} أمراً ضرورياً. يؤكد هذا التطور على الأهمية المتزايدة التي تحظى بها الاستدامة البيئية في قطاع التكنولوجيا.
  
 وقد أدى هذا الوعي إلى اعتماد تدابير استباقية مثل استخدام \LR{Kepler} و \LR{Prometheus} و \LR{Grafana} لمراقبة هذه المقاييس البيئية وتصورها.
 
 يهدف مشروعي الختامي إلى تطبيق \LR{Kepler} لمراقبة استهلاك الطاقة وانبعاثات ثاني أكسيد الكربون في \LR{Kubernetes} و دمجه في سلسلة \LR{CI/CD}.
\end{RLtext}

\noindent\rule[2pt]{\textwidth}{0.5pt}

\begin{RLtext} 

{\textbf{الكلمات المفتاحية:}} \LR{Green IT, DevOps, Kepler, Prometheus, Grafana, Kubernetes, CI/CD}
\\

\end{RLtext}

\noindent\rule[2pt]{\textwidth}{0.5pt}
