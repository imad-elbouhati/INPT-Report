\chapter*{Résumé}
\addcontentsline{toc}{chapter}{Résumé}


Ce rapport reflète le travail réalisé chez Orange Business Maroc dans le cadre de mon projet de fin d'études pour le diplôme d'ingénieur en Télécommunications et Technologies de l'Information.

Dans un contexte où les architectures distribuées et les pratiques de développement agile dominent la conception des applications, intégrer les principes de Green IT dans les pratiques DevOps est devenu essentiel. Cette évolution témoigne de l'importance croissante accordée à la durabilité environnementale dans le secteur de la technologie.

En se concentrant sur la surveillance de la consommation d'énergie et des émissions de CO$_2$ de l'infrastructure, nous mettons en lumière une préoccupation croissante pour l'empreinte écologique de nos systèmes informatiques. Cette prise de conscience a conduit à l'adoption de mesures proactives telles que l'utilisation de Kepler, Prometheus et Grafana pour surveiller et visualiser ces métriques environnementales.

Mon projet de fin d'étude vise à mettre en place Kepler pour surveiller la consommation énergétique et les émissions de CO$_2$ de Kubernetes, et à l'intégrer dans la chaîne CI/CD.

\noindent\rule[2pt]{\textwidth}{0.5pt}

{\textbf{Mots-clés :}}
Green IT, DevOps, Kepler, Prometheus, Grafana, Kubernetes, CI/CD.

\noindent\rule[2pt]{\textwidth}{0.5pt}