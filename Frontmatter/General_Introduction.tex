\chapter*{General Introduction}
\addcontentsline{toc}{chapter}{General Introduction}

Modern applications are typically designed with distributed architectures and developed using agile methodologies. These continuous integration and delivery (CI/CD) practices play a crucial role in efficiently managing the software lifecycle in terms of speed and effectiveness. However, despite this efficiency, security is often overlooked in the CI/CD workflow, particularly regarding concerns related to environmental footprint.

After deploying an application in a real production environment, additional security measures are needed to ensure its protection. The use of Green IT practices, such as integrating metrics for energy consumption and CO$_2$ emissions into the CI/CD pipeline, is becoming increasingly important to ensure the environmental sustainability of technological infrastructures.

My thesis project fits into this context, aiming to automate the monitoring of energy consumption and CO$_2$ emissions in DevOps infrastructures using Kepler. 

\vspace{10pt} % Add space between paragraphs

This report synthesizes the work done throughout my internship period, structured into four chapters covering various aspects of the project. 

\vspace{10pt} 

The first chapter introduces the project context, objectives, and the approach and planning followed. 

\vspace{10pt} 

The second chapter focuses on the project's requirements analysis, while the third chapter addresses the technological choices. 

\vspace{10pt} 

Finally, the fourth chapter presents the project's final structure in detail, along with an overview of the achievements. The report concludes with a summary and a list of bibliographical references.
