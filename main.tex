%%%%%%%%%%%%%%%%%%%%%%%%%%%%%%%%%%%%%%%%%%%%%%%%%%%%%%%%%%%%%%%
%
% Welcome to Overleaf --- just edit your LaTeX on the left,
% and we'll compile it for you on the right. If you open the
% 'Share' menu, you can invite other users to edit at the same
% time. See www.overleaf.com/learn for more info. Enjoy!
%
%%%%%%%%%%%%%%%%%%%%%%%%%%%%%%%%%%%%%%%%%%%%%%%%%%%%%%%%%%%%%%%
\documentclass[12pt,a4paper,oneside]{book}


\usepackage[T1]{fontenc}
\usepackage[utf8]{inputenc}
\usepackage{arabtex}
\usepackage{utf8}
\setcode{utf8}
\usepackage{babel}
\usepackage{svg}
\usepackage{amsmath}
\usepackage{amsthm}
\usepackage{amssymb}
\usepackage{tikz}
\usepackage{tcolorbox}
\usepackage{tabularx}
\usetikzlibrary{calc}
\usepackage[french,ruled,vlined]{algorithm2e}
\usepackage{rotating}
\usepackage{bbding}
\usepackage{cite}
\usepackage{lscape,graphicx}
\usepackage{rotating}
\usepackage{setspace}
\usepackage{sectsty}
\usepackage{dsfont}
\usepackage{ifpdf}
\usepackage{subfigure}
\usepackage{epsfig}
\usepackage{float}
\usepackage{titlesec}
\usepackage{multibib}
\usepackage{fancyhdr}
\usepackage{lipsum}
\usepackage{soul}
\usepackage[paperwidth=210mm,paperheight=297mm,tmargin=15mm,lmargin=20mm]{geometry}
%
\renewcommand{\topfraction}{0.9}
\renewcommand{\textfraction}{0.1}
\renewcommand{\floatpagefraction}{0.8}
%

\usepackage{hyperref}


\urlstyle{same}

\setlength{\headheight}{30pt}
\pagestyle{fancy}% \renewcommand{\chaptermark}[1]{\markboth{#1}{}}
\renewcommand{\chaptermark}[1]{\markboth{\MakeUppercase{#1}}{}}
\renewcommand{\sectionmark}[1]{\markright{\MakeUppercase{\thesection.\ #1}}}


%
% \renewcommand{\sectionmark}[1]{\markright{#1}}
%\renewcommand \thesection{\Roman{section}.}
%\renewcommand \thesubsection{\alph{subsection}.}
%
%\newcommand{\helv}{%
%\fontfamily\fontsize{9}{11}\selectfont}

%

% Clear Header Style on the Last Empty Odd pages
\makeatletter
  \def\cleardoublepage{\clearpage\if@twoside \ifodd\c@page\else%
      \hbox{}%
       \thispagestyle{empty}%              % Empty header styles
       \newpage%
       \if@twocolumn\hbox{}\newpage\fi\fi\fi}
\makeatother


\fancyhead[RE]{\textit{\nouppercase{\leftmark}}}
\fancyhead[LO]{\textit{\nouppercase{\rightmark}}}
\fancyhead[LE,RO]{\thepage}

\renewcommand{\headrulewidth}{0pt}
\renewcommand{\footrulewidth}{0pt}
%
\setcounter{secnumdepth}{4}
\setcounter{tocdepth}{4}
%

%
\def\ds{\displaystyle}
\def\dir{./figures}

%


%\setlength\topmargin{0in}
%\setlength\topmargin{0in}
%

%
% Remove page numbering from first page Bibliography
%
%%%%%%%%%%%%%%%%%%%%%%%%% Page de titre %%%%%%%%%%%%%%%%%%%%%%%
\def\baselinestretch{1.5}


\usepackage{listings}
\usepackage{caption}
\usepackage{longtable}

\usepackage{geometry}
\usepackage{booktabs}

\geometry{
    letterpaper,
    left=1.5cm,
    right=1.5cm,
    top=1.5cm,
    bottom=1.5cm
}

\usepackage{afterpage}

\newcommand\blankpage{%
    \null
    \thispagestyle{empty}%
    \addtocounter{page}{-1}%
    \newpage}


\usepackage[acronym]{glossaries}

\usepackage{lmodern}


\newenvironment{dedication}
  {%\clearpage           % we want a new page          %% I commented this
   \thispagestyle{empty}% no header and footer
   \vspace*{\stretch{1}}% some space at the top
   \itshape             % the text is in italics
   \raggedleft          % flush to the right margin
  }
  {\par % end the paragraph
   \vspace{\stretch{3}} % space at bottom is three times that at the top
   \clearpage           % finish off the page
  }


  



%\makenoidxglossaries 

%\newglossaryentry{latex}
%{
       % name=latex,
        %description={Is a mark up language specially suited for scientific documents}
%}



%\selectlanguage{English}



\hypersetup{
    colorlinks=true,
    linkcolor=black,
    filecolor=magenta,      
    urlcolor=cyan,
    pdfauthor={Name},  %put your name here
    pdftitle={PDF_TTILE},  %PDF title
    pdfpagemode=FullScreen,
    }



\begin{document}



\definecolor{codegreen}{rgb}{0,0.6,0}
    \definecolor{codegray}{rgb}{0.5,0.5,0.5}
    \definecolor{codepurple}{rgb}{0.58,0,0.82}
    \definecolor{backcolour}{rgb}{0.95,0.95,0.92}
    
    \lstdefinestyle{mystyle}{
        backgroundcolor=\color{backcolour},   
        keywordstyle=\color{magenta},
        numberstyle=\tiny\color{codegreen},
        stringstyle=\color{codepurple},
        basicstyle=\ttfamily\footnotesize,
        breakatwhitespace=false,         
        breaklines=true,                 
        captionpos=b,                    
        keepspaces=true,                 
        numbers=left,                    
        numbersep=5pt,                  
        showspaces=false,                
        showstringspaces=false,
        showtabs=false,                  
        tabsize=2
    }




%
\thispagestyle{empty}
\includegraphics[scale=0.08]{Logos/Logo_INPT.png} 
         \hspace{11cm}  
\includegraphics[scale=0.1]{Logos/Logo_ANRT.jpg}
        
\vspace{0.9cm}
\begin{center}
{\large \textsc{\textbf{Mémoire du projet de fin d'études}}}\\[0.1cm]
{\large \textsc{Pour L'obtention du Diplôme d'Ingénieur d'État}}\\[0.1cm]
{\large \textsc{\textit{Filière: XXXXXXXXXXXX}}} \\[0.05cm] 
\vspace{-0.04cm}
% Title
\rule{\linewidth}{0.3mm} \\[0.4cm]   % à ajuster l'éspace en cas de besoin: [1cm]
 { \huge \textbf{ Titre de Projet }} \\[0.15cm] 
\rule{\linewidth}{0.3mm} \\[0.4cm]
\vspace{0.4cm}

\includegraphics[scale=0.075]{Logos/Company_Logo_Expl.png}  %change the scale to suit your logo

\vspace{1cm}

% Author and supervisor
\noindent
\begin{minipage}{0.9\textwidth}
    \vspace{-7mm}
  \begin{flushleft} \large
    \emph{Réalisé par :}\\
    Mme / M. Xxxx \textsc{XXXX} %\& Mme / M. Xxxx \textsc{XXXX}  %Au cas de binôme, remove the % \\
  \end{flushleft}
\end{minipage}
\begin{minipage}{0.4\textwidth}

\end{minipage}\\[0.6cm]

{\large \textit{Soutenu le XX Juillet 20XX, devant le jury composé de : }}\\[0.5cm]


\begin{tabular}{p{1cm}lll}
 & \large M / Mme. Xxxx \textsc{XXXX}  & \large INPT & \large - Examinateur/trice \\[0.1cm]
 & \large M / Mme. Xxxx \textsc{XXXX}  & \large INPT & \large - Examinateur/trice \\[0.1cm]
 & \large M / Mme. Xxxx \textsc{XXXX}  & \large INPT & \large - Encadrant/e \\[0.1cm]
  & \large M / Mme. Xxxx \textsc{XXXX}  & \large Entreprise & \large - Encadrant/e \\[0.1cm]
 
\end{tabular}

\vspace{0.5cm}
\includegraphics[scale=0.6]{Logos/ZLAFA.png}


\textsc{Agence National de Réglementation des Télécommunications}\\
\textsc{Institut National des Postes et Télécommunications}
% Bottom of the page

%\vspace{0.3cm}
{\large Promotion : 20XX/20XX}
   
\end{center}


 %FRENCH ONE


\thispagestyle{empty}
\includegraphics[scale=0.08]{Logos/Logo_INPT.png} 
         \hspace{11cm}  
\includegraphics[scale=0.1]{Logos/Logo_ANRT.jpg}
        
\vspace{0.9cm}
\begin{center}
{\large \textsc{\textbf{Thesis of the end-of-study project}}}\\[0.1cm]
{\large \textsc{To obtain the State Engineering Diploma}}\\[0.1cm]
{\large \textsc{\textit{Major: XXXXXXXXXXXX}}} \\[0.05cm] 
\vspace{-0.04cm}
% Title
\rule{\linewidth}{0.3mm} \\[0.4cm]   % à ajuster l'éspace en cas de besoin: [1cm]
 { \huge \textbf{ Project Title }} \\[0.15cm] 
\rule{\linewidth}{0.3mm} \\[0.4cm]
\vspace{0.4cm}

\includegraphics[scale=0.075]{Logos/Company_Logo_Expl.png}  %change the scale to suit your logo

\vspace{1cm}

% Author and supervisor
\noindent
\begin{minipage}{0.9\textwidth}
    \vspace{-7mm}
  \begin{flushleft} \large
    \emph{Authored by :}\\
    Mr/Mrs/Ms. Xxxx \textsc{XXXX} %\& Mr/Mrs/Ms. Xxxx \textsc{XXXX}  %Au cas de binôme, remove the % \\
  \end{flushleft}
\end{minipage}
\begin{minipage}{0.4\textwidth}

\end{minipage}\\[0.6cm]

{\large \textit{Defended on July XX, 20XX, before the jury composed of :}}\\[0.5cm]


\begin{tabular}{p{1cm}lll}
 & \large Mr / Mrs. Xxxx \textsc{XXXX}  & \large INPT & \large - Examiner \\[0.1cm]
 & \large Mr / Mrs. Xxxx \textsc{XXXX}  & \large INPT & \large - Examiner \\[0.1cm]
 & \large Mr / Mrs. Xxxx \textsc{XXXX}  & \large INPT & \large - Supervisor \\[0.1cm]
  & \large Mr / Mrs. Xxxx \textsc{XXXX}  & \large "Company" & \large - Supervisor \\[0.1cm]
 
\end{tabular}

\vspace{0.5cm}
\includegraphics[scale=0.6]{Logos/ZLAFA.png}


\textsc{Agence National de Réglementation des Télécommunications}\\
\textsc{Institut National des Postes et Télécommunications}
% Bottom of the page

%\vspace{0.3cm}
{\large Class of : 20XX/20XX}
   
\end{center}


 % ENGLISH

\afterpage{\blankpage}  %page vide obligatoire

\frontmatter


\chapter*{Dedicated to}
\addcontentsline{toc}{chapter}{Dedication}

\begin{dedication}
  
  \hspace{1em} I humbly dedicate this work to my dear mother, who gave me life, love, courage, and a reason to live.
  I also wish to express my gratitude to my father for his support and sacrifices.
  \vspace{\baselineskip}
  \par   %% or a blank line
  
  A big thank you to my two dear sisters, Alae and Samia, whom I love so much,
  to all my friends with whom I have shared moments of joy and happiness, as well as to my entire extended family and to all those who love me.

  \vspace{\baselineskip}
  \hspace*{\fill} \usefont{T1}{LobsterTwo-LF}{bx}{it} Imad
  
\end{dedication}


\chapter*{Acknowledgments}
\addcontentsline{toc}{chapter}{Acknowledgments}


First and foremost, I would like to express my gratitude to Allah the Almighty for giving me the strength and patience necessary to complete this project.

I would like to extend my sincerest thanks to my supervisor, Mrs. Charifa HANIN, for her competent assistance, patience, and encouragement. Her critical feedback has been invaluable in structuring my work and improving the quality of its various sections.

I also want to express my appreciation to my mentor, Mr. AJA Amine, for his immense support, the quality of his guidance, and for all the advice and information he provided with unmatched professionalism and patience.

A big thank you is also owed to Mr. MOUSTACHI Mouhsine, for offering me the opportunity to join his team and for his support.

I would also like to thank Mr. EDDAHBY Mohamed Amine, Mr. ECHARJI Abderrahman, Mr. OUCHEN Othman, Mr. EL ALJ Mohamed Taieb, Mr. EL IDRISSI Hicham, Ms. EL IBRAHIMI Dounia, Mr. OUADHDHAFE Marouane, and all the engineers in the DevOps team for their invaluable assistance, encouragement, and for making my internship at Orange Business a very enriching experience.

I wish to express my sincere gratitude to the members of the jury for the honor they bestow upon me by taking the time to read and evaluate my work.

I would also like to thank the pedagogical and administrative team at INPT for their dedication and contribution to our excellent education.

Finally, I would like to thank all the individuals who have contributed, directly or indirectly, to the completion of this work.


\chapter*{Résumé}
\addcontentsline{toc}{chapter}{Résumé}


Ce rapport reflète le travail réalisé chez Orange Business Maroc dans le cadre de mon projet de fin d'études pour le diplôme d'ingénieur en Télécommunications et Technologies de l'Information.

Dans un contexte où les architectures distribuées et les pratiques de développement agile dominent la conception des applications, intégrer les principes de Green IT dans les pratiques DevOps est devenu essentiel. Cette évolution témoigne de l'importance croissante accordée à la durabilité environnementale dans le secteur de la technologie.

En se concentrant sur la surveillance de la consommation d'énergie et des émissions de CO$_2$ de l'infrastructure, nous mettons en lumière une préoccupation croissante pour l'empreinte écologique de nos systèmes informatiques. Cette prise de conscience a conduit à l'adoption de mesures proactives telles que l'utilisation de Kepler, Prometheus et Grafana pour surveiller et visualiser ces métriques environnementales.

Mon projet de fin d'étude vise à mettre en place Kepler pour surveiller la consommation énergétique et les émissions de CO$_2$ de Kubernetes, et à l'intégrer dans la chaîne CI/CD.

\noindent\rule[2pt]{\textwidth}{0.5pt}

{\textbf{Mots-clés :}}
Green IT, DevOps, Kepler, Prometheus, Grafana, Kubernetes, CI/CD.

\noindent\rule[2pt]{\textwidth}{0.5pt}

% \cleardoublepage
%

\chapter*{Summary}
\addcontentsline{toc}{chapter}{Summary}


Lorem ipsum dolor sit amet, consectetur adipiscing elit. Praesent nec dapibus justo. Donec sagittis vulputate ante sed porttitor. Suspendisse sit amet nisl massa. Curabitur nec nisl condimentum, egestas ex vitae, dapibus enim. Etiam iaculis, erat faucibus pellentesque sagittis, nisi justo sollicitudin nibh, et condimentum augue massa non turpis. Proin commodo enim fermentum suscipit condimentum. Maecenas molestie, dui nec vestibulum rhoncus, arcu nisl faucibus neque, a ornare nisi massa ac eros. Aenean id velit sit amet lacus mattis varius. Donec fringilla massa sed nisi eleifend, a aliquet mi tempus. Nunc posuere euismod est, nec tristique augue lobortis non. Sed sodales sem ut metus tempus ullamcorper.

\noindent\rule[2pt]{\textwidth}{0.5pt}

{\textbf{Key Words :}}
xxxx, xxxx, xxx, xxx.
\\
\noindent\rule[2pt]{\textwidth}{0.5pt}

% \cleardoublepage
%

\chapter*{\RL{ملخص}}
\addcontentsline{toc}{chapter}{Arabic Abstract}

\begin{RLtext}
يعكس هذا التقرير العمل الذي قمت به في أورانج بيزنس المغرب في إطار مشروعي الختامي للحصول على شهادة مهندس دولة في الاتصالات وتكنولوجيا المعلومات.

 في سياق تهيمن فيه الهياكل الموزعة وممارسات التطوير السريعة على تصميم التطبيقات، أصبح دمج مبادئ تكنولوجيا الخضراء في ممارسات \LR{DevOps} أمراً ضرورياً. يؤكد هذا التطور على الأهمية المتزايدة التي تحظى بها الاستدامة البيئية في قطاع التكنولوجيا.
  
 وقد أدى هذا الوعي إلى اعتماد تدابير استباقية مثل استخدام \LR{Kepler} و \LR{Prometheus} و \LR{Grafana} لمراقبة هذه المقاييس البيئية وتصورها.
 
 يهدف مشروعي الختامي إلى تطبيق \LR{Kepler} لمراقبة استهلاك الطاقة وانبعاثات ثاني أكسيد الكربون في \LR{Kubernetes} و دمجه في سلسلة \LR{CI/CD}.
\end{RLtext}

\noindent\rule[2pt]{\textwidth}{0.5pt}

\begin{RLtext} 

{\textbf{الكلمات المفتاحية:}} \LR{Green IT, DevOps, Kepler, Prometheus, Grafana, Kubernetes, CI/CD}
\\

\end{RLtext}

\noindent\rule[2pt]{\textwidth}{0.5pt}


% \cleardoublepage
%






\listoffigures
\addcontentsline{toc}{chapter}{List of Figures} 
%figures are added automatically here

\listoftables
\addcontentsline{toc}{chapter}{List of Tables} 
%tables are added automatically here

\lstlistoflistings
\addcontentsline{toc}{chapter}{List of Listings} 
%Code snippets are added automatically here

\tableofcontents
\addcontentsline{toc}{chapter}{Table of Contents}
%contents are added automatically here

\mainmatter

%debut of chapters

\chapter{General Project Context}
\label{chap:General Project Context}

% "*" makes the section unnumbered

\section*{Introduction}

Use this template as you wish, change what you want to change, the section titles are only examples, you don't have to follow them to the letter.


This is an example of me citing the 1st reference in the bibliography at the end of this report \cite{ref1}. Use it well!

The next section contains the README text that's also found in the left part along with the other files.


\newpage

\section{READ\_ME}

Hi! 

This template is a combination of multiple student and teacher PFE report templates that I have compiled into one that hopefully will satisfy your needs.
\\

It is in English, but I have included the french "Page de garde" if you want to use it, and the rest of the paper is easily translatable.
\\

This document is compiled using pdfLatex Compiler, so make sure you select it in the menu on the top left of the page. You can change the font size there along with other things.
\\

Some table, figure, list or formatting codes can be found in the "Codes\_needed.tex" file in this same folder, use them well.
\\

The organisation of this template is as follows: 
\\
The main compilation file is main.tex, any file you want to add, should be added there using, %\chapter{General Project Context}
\label{chap:General Project Context}

% "*" makes the section unnumbered

\section*{Introduction}

Use this template as you wish, change what you want to change, the section titles are only examples, you don't have to follow them to the letter.


This is an example of me citing the 1st reference in the bibliography at the end of this report \cite{ref1}. Use it well!

The next section contains the README text that's also found in the left part along with the other files.


\newpage

\section{READ\_ME}

Hi! 

This template is a combination of multiple student and teacher PFE report templates that I have compiled into one that hopefully will satisfy your needs.
\\

It is in English, but I have included the french "Page de garde" if you want to use it, and the rest of the paper is easily translatable.
\\

This document is compiled using pdfLatex Compiler, so make sure you select it in the menu on the top left of the page. You can change the font size there along with other things.
\\

Some table, figure, list or formatting codes can be found in the "Codes\_needed.tex" file in this same folder, use them well.
\\

The organisation of this template is as follows: 
\\
The main compilation file is main.tex, any file you want to add, should be added there using, %\chapter{General Project Context}
\label{chap:General Project Context}

% "*" makes the section unnumbered

\section*{Introduction}

Use this template as you wish, change what you want to change, the section titles are only examples, you don't have to follow them to the letter.


This is an example of me citing the 1st reference in the bibliography at the end of this report \cite{ref1}. Use it well!

The next section contains the README text that's also found in the left part along with the other files.


\newpage

\section{READ\_ME}

Hi! 

This template is a combination of multiple student and teacher PFE report templates that I have compiled into one that hopefully will satisfy your needs.
\\

It is in English, but I have included the french "Page de garde" if you want to use it, and the rest of the paper is easily translatable.
\\

This document is compiled using pdfLatex Compiler, so make sure you select it in the menu on the top left of the page. You can change the font size there along with other things.
\\

Some table, figure, list or formatting codes can be found in the "Codes\_needed.tex" file in this same folder, use them well.
\\

The organisation of this template is as follows: 
\\
The main compilation file is main.tex, any file you want to add, should be added there using, %\input{Chapters/Chapter1} for example. 

Remember to change the PDF Title and author name before the begin document command.
\\

Packages.tex is where you import packages and could modify their options.
\\

The frontmatter folder contains unnumbered chapters that come before the actual chapters, so the resumes and acknowledgments are there. The pages are numbered in Roman numbers.
\\

The chapters folder obviously contains the main chapters of the report, usually the first one is an intro, of both the project and the company, the last one is a conclusion chapter, I made it unnumbered here but you do you.
\\

The endmatter folder contains the appendices, acronyms, glossary, and Complementary figures, tables and codes. Consider checking this link \url{https://libguides.usc.edu/writingguide/appendices} for more info. Usually you add an appendix for each subject you'll talk about it, each with its own codes, tables, figures and text.
\\

The bibliography can be found at the end of main.tex file.
\\

And to organise your figures better, upload the logos to the logos folder, and content related figures should go in the figures folder, where you can add sub folders.
\\

Along the template, make sure to read my comments, they can be helpful to understand the purpose of a command or option. 
\\

When you finish writing your thesis, make sure to verify that you didn't leave any generic line or link. Revise it well.
\\

There are 10 warnings that show up in this template, some I couldn't manage to solve (or understand), and some I left since they are necessary for what I intend of this template.
\\

Obviously this template is only a suggestion, it is not perfect in any sense, you can improve it in the way that suits you, so search away, and get used to reading the documentation.
\\

Also consult with your supervisor, as each teacher has their own opinion on what constitutes the ideal report.
\\

Finally, I hope you have enjoyed your time at INPT as much as I did, and Good Luck :D
\\

-Mery


\subsection{Codes\_Needed}

This subsection includes codes for different elements you will need: figures, tables, lists...

Copy the codes you want and test them in the chapter files.

if you want symbols and other text styles, visit this link: 

\href{https://www.cmor-faculty.rice.edu/~heinken/latex/symbols.pdf}{Symbols}

Read the comments !!

% Content division

%\chapter{Comes first}, then \section{}, then \subsection{}, then \subsubsection{}.

\subsubsection{Text formatting}

\textbf{This text is bold}

\textit{This text is italic}

\underline{This text is underlined.}

\st{This text is struck out.}

\textsc{This text is capitalized.}

%Use \paragraph{To start a paragraph}


Some characters like "\%", "\$" and "\&" are significant in Latex code, so to include them in normal text, use the backslash character before them.
To print out backslash, use \symbol{92}


Documentation: \href{https://www.overleaf.com/learn/latex/Bold%2C_italics_and_underlining}{Italics and underlining}


\subsubsection{Figures} 

\begin{figure}[H] 
    \centering
    \includegraphics[width=4cm]{Logos/Logo_INPT.png}
    \caption{Caption}
    \label{fig:my_label} %Optional (If you want to reference the figure in later chapters)
\end{figure}

%[width=7cm] you control the size of the image. other options include: 
%[height=7cm] or [scale=0.5] (means half the size of the original image)


Documentation: \href{https://www.overleaf.com/learn/latex/Inserting_Images}{Images}


\subsubsection{Tables} 

Simple table without borders:
\\

\begin{tabular}{ll}
  First & Second \\
  Third & Fourth
\end{tabular}
\\

More complex table with borders:
\\

\begin{tabular}{|l|c|r|} \hline
  Left aligned column & Centered column & Right aligned column \\ \hline
  Text & Text & Text \\ \hline
\end{tabular}
\\

Example of a short table

%{5cm} is the cell length, you can change it to suit your own table

\begin{table}[H]
    \centering
    \begin{tabular}{|m{5cm}|m{10cm}|}
        \hline
          Column1 & Column2 \\
        \hline
          Element11 & Element21 \\
        \hline
          Element12 & Element22 \\
        \hline
          Element13 & Element23 \\
        \hline
    \end{tabular}
    \caption{Table Example}
\end{table}


Example of a long table (that spans 2 pages or more), Latex will automatically split the table when it reaches the end of the page:

\begin{longtable}[c]{| m{4.4cm} | m{11cm} |}
\caption{Long table}\\
 \hline

 Cell & Description  \\ 
 \hline
 \endfirsthead

 \hline
 
 Cell & Description  \\ 
 \hline
 \endhead

        \hline
          Element11 & Element21 \\
        \hline
          Element12 & Element22 \\
        \hline
          Element13 & Element23 \\
        \hline
          Element14 & Element24 \\
        \hline
          Element15 & Element25 \\
        \hline
          Element16 & Element26 \\
        \hline
          Element17 & Element27 \\
        \hline
          Element18 & Element28 \\
        \hline
          Element19 & Element29 \\
        \hline
          Element110 & Element210 \\
        \hline
          Element111 & Element211 \\
        \hline
          Element112 & Element212 \\
        \hline
          Element113 & Element213 \\
        \hline
          Element114 & Element214 \\
        \hline

 \end{longtable}


Documentation: \href{https://www.overleaf.com/learn/latex/Tables}{Tables}


\subsubsection{Lists}

To start an unnumbered list, use:

\begin{itemize}
    \item 
    \item 
    \item 
\end{itemize}

To start a numbered list, use:

\begin{enumerate}
    \item 
    \item 
    \item 
\end{enumerate}



Documentation: \href{https://www.overleaf.com/learn/latex/Lists}{Lists}


\subsubsection{Code scripts or terminal}

Say you have a script or terminal command you want to include, you use the following code:

    \lstset{style=mystyle} %this style is already defined in Packages.tex
    
    \begin{lstlisting}[language=bash, caption= Code caption]
    
    root@eve-ng:~# mkdir -p /opt/unetlab/addons/qemu/timos-20.10.R12

    \end{lstlisting}


Documentation: \href{https://www.overleaf.com/learn/latex/Code_listing}{Code Listing}

\subsubsection{Math}

Some math formulas for you, test them in your chapters:

These are inline formulas: $x$, $a_i^2 + b_i^2 \le a_{i+1}^2$. Afterwards...

These are centered formulas: $$x,$$ $$a_i^2 + b_i^2 \le a_{i+1}^2.$$ Afterwards...

Some complex formula: $$P(|S - E[S]| \ge t) \le 2 \exp \left( -\frac{2 t^2 n^2}{\sum_{i = 1}^n (b_i - a_i)^2} \right).$$

Also you can use the first link for math symbols and other useful stuff:

Documentation: \href{https://www.cmor-faculty.rice.edu/~heinken/latex/symbols.pdf}{Symbols file again}



\newpage


\section{Presentation of host organization}



\subsection{Company Overview}

\begin{figure}[H] 
    \centering
    \includegraphics[width=7cm]{Logos/Company_Logo_Expl.png}
    \caption{Company logo}
    %\label{fig:my_label} %Optional (If you want to reference the figure in later chapters)
\end{figure}



\subsection{Organizational Chart}


\begin{figure}[H] 
    \centering
    \includegraphics[width=12cm]{Figures/Organizational_Chart.png}
    \caption{Organizational Chart}
    %\label{fig:my_label} %Optional (If you want to reference the figure in later chapters)
\end{figure}










\section{Presentation of the project}


\subsection{Project Framework}



\subsection{Project objectives}





\subsection{Project Planning}

\begin{figure}[H] 
    \centering
    \includegraphics[width=12cm]{Figures/Gantt_Diagram.png}
    \caption{Gantt Diagram}
    %\label{fig:my_label} %Optional (If you want to reference the figure in later chapters)
\end{figure}

\newpage

\section*{Conclusion}

Lorem ipsum dolor sit amet, consectetur adipiscing elit. Praesent nec dapibus justo. Donec sagittis vulputate ante sed porttitor. Suspendisse sit amet nisl massa. Curabitur nec nisl condimentum, egestas ex vitae, dapibus enim. Etiam iaculis, erat faucibus pellentesque sagittis, nisi justo sollicitudin nibh, et condimentum augue massa non turpis. Proin commodo enim fermentum suscipit condimentum. Maecenas molestie, dui nec vestibulum rhoncus, arcu nisl faucibus neque, a ornare nisi massa ac eros. Aenean id velit sit amet lacus mattis varius. Donec fringilla massa sed nisi eleifend, a aliquet mi tempus. Nunc posuere euismod est, nec tristique augue lobortis non. Sed sodales sem ut metus tempus ullamcorper.
 for example. 

Remember to change the PDF Title and author name before the begin document command.
\\

Packages.tex is where you import packages and could modify their options.
\\

The frontmatter folder contains unnumbered chapters that come before the actual chapters, so the resumes and acknowledgments are there. The pages are numbered in Roman numbers.
\\

The chapters folder obviously contains the main chapters of the report, usually the first one is an intro, of both the project and the company, the last one is a conclusion chapter, I made it unnumbered here but you do you.
\\

The endmatter folder contains the appendices, acronyms, glossary, and Complementary figures, tables and codes. Consider checking this link \url{https://libguides.usc.edu/writingguide/appendices} for more info. Usually you add an appendix for each subject you'll talk about it, each with its own codes, tables, figures and text.
\\

The bibliography can be found at the end of main.tex file.
\\

And to organise your figures better, upload the logos to the logos folder, and content related figures should go in the figures folder, where you can add sub folders.
\\

Along the template, make sure to read my comments, they can be helpful to understand the purpose of a command or option. 
\\

When you finish writing your thesis, make sure to verify that you didn't leave any generic line or link. Revise it well.
\\

There are 10 warnings that show up in this template, some I couldn't manage to solve (or understand), and some I left since they are necessary for what I intend of this template.
\\

Obviously this template is only a suggestion, it is not perfect in any sense, you can improve it in the way that suits you, so search away, and get used to reading the documentation.
\\

Also consult with your supervisor, as each teacher has their own opinion on what constitutes the ideal report.
\\

Finally, I hope you have enjoyed your time at INPT as much as I did, and Good Luck :D
\\

-Mery


\subsection{Codes\_Needed}

This subsection includes codes for different elements you will need: figures, tables, lists...

Copy the codes you want and test them in the chapter files.

if you want symbols and other text styles, visit this link: 

\href{https://www.cmor-faculty.rice.edu/~heinken/latex/symbols.pdf}{Symbols}

Read the comments !!

% Content division

%\chapter{Comes first}, then \section{}, then \subsection{}, then \subsubsection{}.

\subsubsection{Text formatting}

\textbf{This text is bold}

\textit{This text is italic}

\underline{This text is underlined.}

\st{This text is struck out.}

\textsc{This text is capitalized.}

%Use \paragraph{To start a paragraph}


Some characters like "\%", "\$" and "\&" are significant in Latex code, so to include them in normal text, use the backslash character before them.
To print out backslash, use \symbol{92}


Documentation: \href{https://www.overleaf.com/learn/latex/Bold%2C_italics_and_underlining}{Italics and underlining}


\subsubsection{Figures} 

\begin{figure}[H] 
    \centering
    \includegraphics[width=4cm]{Logos/Logo_INPT.png}
    \caption{Caption}
    \label{fig:my_label} %Optional (If you want to reference the figure in later chapters)
\end{figure}

%[width=7cm] you control the size of the image. other options include: 
%[height=7cm] or [scale=0.5] (means half the size of the original image)


Documentation: \href{https://www.overleaf.com/learn/latex/Inserting_Images}{Images}


\subsubsection{Tables} 

Simple table without borders:
\\

\begin{tabular}{ll}
  First & Second \\
  Third & Fourth
\end{tabular}
\\

More complex table with borders:
\\

\begin{tabular}{|l|c|r|} \hline
  Left aligned column & Centered column & Right aligned column \\ \hline
  Text & Text & Text \\ \hline
\end{tabular}
\\

Example of a short table

%{5cm} is the cell length, you can change it to suit your own table

\begin{table}[H]
    \centering
    \begin{tabular}{|m{5cm}|m{10cm}|}
        \hline
          Column1 & Column2 \\
        \hline
          Element11 & Element21 \\
        \hline
          Element12 & Element22 \\
        \hline
          Element13 & Element23 \\
        \hline
    \end{tabular}
    \caption{Table Example}
\end{table}


Example of a long table (that spans 2 pages or more), Latex will automatically split the table when it reaches the end of the page:

\begin{longtable}[c]{| m{4.4cm} | m{11cm} |}
\caption{Long table}\\
 \hline

 Cell & Description  \\ 
 \hline
 \endfirsthead

 \hline
 
 Cell & Description  \\ 
 \hline
 \endhead

        \hline
          Element11 & Element21 \\
        \hline
          Element12 & Element22 \\
        \hline
          Element13 & Element23 \\
        \hline
          Element14 & Element24 \\
        \hline
          Element15 & Element25 \\
        \hline
          Element16 & Element26 \\
        \hline
          Element17 & Element27 \\
        \hline
          Element18 & Element28 \\
        \hline
          Element19 & Element29 \\
        \hline
          Element110 & Element210 \\
        \hline
          Element111 & Element211 \\
        \hline
          Element112 & Element212 \\
        \hline
          Element113 & Element213 \\
        \hline
          Element114 & Element214 \\
        \hline

 \end{longtable}


Documentation: \href{https://www.overleaf.com/learn/latex/Tables}{Tables}


\subsubsection{Lists}

To start an unnumbered list, use:

\begin{itemize}
    \item 
    \item 
    \item 
\end{itemize}

To start a numbered list, use:

\begin{enumerate}
    \item 
    \item 
    \item 
\end{enumerate}



Documentation: \href{https://www.overleaf.com/learn/latex/Lists}{Lists}


\subsubsection{Code scripts or terminal}

Say you have a script or terminal command you want to include, you use the following code:

    \lstset{style=mystyle} %this style is already defined in Packages.tex
    
    \begin{lstlisting}[language=bash, caption= Code caption]
    
    root@eve-ng:~# mkdir -p /opt/unetlab/addons/qemu/timos-20.10.R12

    \end{lstlisting}


Documentation: \href{https://www.overleaf.com/learn/latex/Code_listing}{Code Listing}

\subsubsection{Math}

Some math formulas for you, test them in your chapters:

These are inline formulas: $x$, $a_i^2 + b_i^2 \le a_{i+1}^2$. Afterwards...

These are centered formulas: $$x,$$ $$a_i^2 + b_i^2 \le a_{i+1}^2.$$ Afterwards...

Some complex formula: $$P(|S - E[S]| \ge t) \le 2 \exp \left( -\frac{2 t^2 n^2}{\sum_{i = 1}^n (b_i - a_i)^2} \right).$$

Also you can use the first link for math symbols and other useful stuff:

Documentation: \href{https://www.cmor-faculty.rice.edu/~heinken/latex/symbols.pdf}{Symbols file again}



\newpage


\section{Presentation of host organization}



\subsection{Company Overview}

\begin{figure}[H] 
    \centering
    \includegraphics[width=7cm]{Logos/Company_Logo_Expl.png}
    \caption{Company logo}
    %\label{fig:my_label} %Optional (If you want to reference the figure in later chapters)
\end{figure}



\subsection{Organizational Chart}


\begin{figure}[H] 
    \centering
    \includegraphics[width=12cm]{Figures/Organizational_Chart.png}
    \caption{Organizational Chart}
    %\label{fig:my_label} %Optional (If you want to reference the figure in later chapters)
\end{figure}










\section{Presentation of the project}


\subsection{Project Framework}



\subsection{Project objectives}





\subsection{Project Planning}

\begin{figure}[H] 
    \centering
    \includegraphics[width=12cm]{Figures/Gantt_Diagram.png}
    \caption{Gantt Diagram}
    %\label{fig:my_label} %Optional (If you want to reference the figure in later chapters)
\end{figure}

\newpage

\section*{Conclusion}

Lorem ipsum dolor sit amet, consectetur adipiscing elit. Praesent nec dapibus justo. Donec sagittis vulputate ante sed porttitor. Suspendisse sit amet nisl massa. Curabitur nec nisl condimentum, egestas ex vitae, dapibus enim. Etiam iaculis, erat faucibus pellentesque sagittis, nisi justo sollicitudin nibh, et condimentum augue massa non turpis. Proin commodo enim fermentum suscipit condimentum. Maecenas molestie, dui nec vestibulum rhoncus, arcu nisl faucibus neque, a ornare nisi massa ac eros. Aenean id velit sit amet lacus mattis varius. Donec fringilla massa sed nisi eleifend, a aliquet mi tempus. Nunc posuere euismod est, nec tristique augue lobortis non. Sed sodales sem ut metus tempus ullamcorper.
 for example. 

Remember to change the PDF Title and author name before the begin document command.
\\

Packages.tex is where you import packages and could modify their options.
\\

The frontmatter folder contains unnumbered chapters that come before the actual chapters, so the resumes and acknowledgments are there. The pages are numbered in Roman numbers.
\\

The chapters folder obviously contains the main chapters of the report, usually the first one is an intro, of both the project and the company, the last one is a conclusion chapter, I made it unnumbered here but you do you.
\\

The endmatter folder contains the appendices, acronyms, glossary, and Complementary figures, tables and codes. Consider checking this link \url{https://libguides.usc.edu/writingguide/appendices} for more info. Usually you add an appendix for each subject you'll talk about it, each with its own codes, tables, figures and text.
\\

The bibliography can be found at the end of main.tex file.
\\

And to organise your figures better, upload the logos to the logos folder, and content related figures should go in the figures folder, where you can add sub folders.
\\

Along the template, make sure to read my comments, they can be helpful to understand the purpose of a command or option. 
\\

When you finish writing your thesis, make sure to verify that you didn't leave any generic line or link. Revise it well.
\\

There are 10 warnings that show up in this template, some I couldn't manage to solve (or understand), and some I left since they are necessary for what I intend of this template.
\\

Obviously this template is only a suggestion, it is not perfect in any sense, you can improve it in the way that suits you, so search away, and get used to reading the documentation.
\\

Also consult with your supervisor, as each teacher has their own opinion on what constitutes the ideal report.
\\

Finally, I hope you have enjoyed your time at INPT as much as I did, and Good Luck :D
\\

-Mery


\subsection{Codes\_Needed}

This subsection includes codes for different elements you will need: figures, tables, lists...

Copy the codes you want and test them in the chapter files.

if you want symbols and other text styles, visit this link: 

\href{https://www.cmor-faculty.rice.edu/~heinken/latex/symbols.pdf}{Symbols}

Read the comments !!

% Content division

%\chapter{Comes first}, then \section{}, then \subsection{}, then \subsubsection{}.

\subsubsection{Text formatting}

\textbf{This text is bold}

\textit{This text is italic}

\underline{This text is underlined.}

\st{This text is struck out.}

\textsc{This text is capitalized.}

%Use \paragraph{To start a paragraph}


Some characters like "\%", "\$" and "\&" are significant in Latex code, so to include them in normal text, use the backslash character before them.
To print out backslash, use \symbol{92}


Documentation: \href{https://www.overleaf.com/learn/latex/Bold%2C_italics_and_underlining}{Italics and underlining}


\subsubsection{Figures} 

\begin{figure}[H] 
    \centering
    \includegraphics[width=4cm]{Logos/Logo_INPT.png}
    \caption{Caption}
    \label{fig:my_label} %Optional (If you want to reference the figure in later chapters)
\end{figure}

%[width=7cm] you control the size of the image. other options include: 
%[height=7cm] or [scale=0.5] (means half the size of the original image)


Documentation: \href{https://www.overleaf.com/learn/latex/Inserting_Images}{Images}


\subsubsection{Tables} 

Simple table without borders:
\\

\begin{tabular}{ll}
  First & Second \\
  Third & Fourth
\end{tabular}
\\

More complex table with borders:
\\

\begin{tabular}{|l|c|r|} \hline
  Left aligned column & Centered column & Right aligned column \\ \hline
  Text & Text & Text \\ \hline
\end{tabular}
\\

Example of a short table

%{5cm} is the cell length, you can change it to suit your own table

\begin{table}[H]
    \centering
    \begin{tabular}{|m{5cm}|m{10cm}|}
        \hline
          Column1 & Column2 \\
        \hline
          Element11 & Element21 \\
        \hline
          Element12 & Element22 \\
        \hline
          Element13 & Element23 \\
        \hline
    \end{tabular}
    \caption{Table Example}
\end{table}


Example of a long table (that spans 2 pages or more), Latex will automatically split the table when it reaches the end of the page:

\begin{longtable}[c]{| m{4.4cm} | m{11cm} |}
\caption{Long table}\\
 \hline

 Cell & Description  \\ 
 \hline
 \endfirsthead

 \hline
 
 Cell & Description  \\ 
 \hline
 \endhead

        \hline
          Element11 & Element21 \\
        \hline
          Element12 & Element22 \\
        \hline
          Element13 & Element23 \\
        \hline
          Element14 & Element24 \\
        \hline
          Element15 & Element25 \\
        \hline
          Element16 & Element26 \\
        \hline
          Element17 & Element27 \\
        \hline
          Element18 & Element28 \\
        \hline
          Element19 & Element29 \\
        \hline
          Element110 & Element210 \\
        \hline
          Element111 & Element211 \\
        \hline
          Element112 & Element212 \\
        \hline
          Element113 & Element213 \\
        \hline
          Element114 & Element214 \\
        \hline

 \end{longtable}


Documentation: \href{https://www.overleaf.com/learn/latex/Tables}{Tables}


\subsubsection{Lists}

To start an unnumbered list, use:

\begin{itemize}
    \item 
    \item 
    \item 
\end{itemize}

To start a numbered list, use:

\begin{enumerate}
    \item 
    \item 
    \item 
\end{enumerate}



Documentation: \href{https://www.overleaf.com/learn/latex/Lists}{Lists}


\subsubsection{Code scripts or terminal}

Say you have a script or terminal command you want to include, you use the following code:

    \lstset{style=mystyle} %this style is already defined in Packages.tex
    
    \begin{lstlisting}[language=bash, caption= Code caption]
    
    root@eve-ng:~# mkdir -p /opt/unetlab/addons/qemu/timos-20.10.R12

    \end{lstlisting}


Documentation: \href{https://www.overleaf.com/learn/latex/Code_listing}{Code Listing}

\subsubsection{Math}

Some math formulas for you, test them in your chapters:

These are inline formulas: $x$, $a_i^2 + b_i^2 \le a_{i+1}^2$. Afterwards...

These are centered formulas: $$x,$$ $$a_i^2 + b_i^2 \le a_{i+1}^2.$$ Afterwards...

Some complex formula: $$P(|S - E[S]| \ge t) \le 2 \exp \left( -\frac{2 t^2 n^2}{\sum_{i = 1}^n (b_i - a_i)^2} \right).$$

Also you can use the first link for math symbols and other useful stuff:

Documentation: \href{https://www.cmor-faculty.rice.edu/~heinken/latex/symbols.pdf}{Symbols file again}



\newpage


\section{Presentation of host organization}



\subsection{Company Overview}

\begin{figure}[H] 
    \centering
    \includegraphics[width=7cm]{Logos/Company_Logo_Expl.png}
    \caption{Company logo}
    %\label{fig:my_label} %Optional (If you want to reference the figure in later chapters)
\end{figure}



\subsection{Organizational Chart}


\begin{figure}[H] 
    \centering
    \includegraphics[width=12cm]{Figures/Organizational_Chart.png}
    \caption{Organizational Chart}
    %\label{fig:my_label} %Optional (If you want to reference the figure in later chapters)
\end{figure}










\section{Presentation of the project}


\subsection{Project Framework}



\subsection{Project objectives}





\subsection{Project Planning}

\begin{figure}[H] 
    \centering
    \includegraphics[width=12cm]{Figures/Gantt_Diagram.png}
    \caption{Gantt Diagram}
    %\label{fig:my_label} %Optional (If you want to reference the figure in later chapters)
\end{figure}

\newpage

\section*{Conclusion}

Lorem ipsum dolor sit amet, consectetur adipiscing elit. Praesent nec dapibus justo. Donec sagittis vulputate ante sed porttitor. Suspendisse sit amet nisl massa. Curabitur nec nisl condimentum, egestas ex vitae, dapibus enim. Etiam iaculis, erat faucibus pellentesque sagittis, nisi justo sollicitudin nibh, et condimentum augue massa non turpis. Proin commodo enim fermentum suscipit condimentum. Maecenas molestie, dui nec vestibulum rhoncus, arcu nisl faucibus neque, a ornare nisi massa ac eros. Aenean id velit sit amet lacus mattis varius. Donec fringilla massa sed nisi eleifend, a aliquet mi tempus. Nunc posuere euismod est, nec tristique augue lobortis non. Sed sodales sem ut metus tempus ullamcorper.


\chapter{Specification and Needs Analysis}
\label{chap:Chapter 2 title}
\section{Introduction}


The success of a project heavily relies on the quality of its initiation. Therefore, the functional study phase is crucial for a successful start to the project. This chapter encompasses requirement specifications, including needs analysis involving stakeholder identification, system use cases, and the textual scenarios associated with these use cases.

\pagebreak

\section{Study of the Current State}

Orange Business Services Morocco has embraced an approach known as CI/CD to efficiently manage the development and delivery of its projects. They utilize the GitLab platform, which facilitates swift creation, compilation, testing, delivery, and deployment of software.

In their existing solution, they have established continuous integration and delivery pipelines using GitLab CI, Nexus, Docker, and Kubernetes. Here's how it operates:

\begin{itemize}
  \item Code management and updates are performed using GitLab. Developers push their code changes to the GitLab repository.
  \item The application is containerized using Docker, a technology enabling the creation of isolated and portable environments for applications.
  \item The GitLab Runner conducts a check to ensure the YAML code is correct and clean. Then, it triggers pipeline steps such as Docker image generation and their submission to the Nexus repository.
  \item Kubernetes, aided by Helm (a package manager for Kubernetes), handles the application deployment across various environments. It follows a "Push" deployment model.
\end{itemize}

This approach enables Orange Business Services Morocco to efficiently manage the development and delivery of their projects by automating processes using GitLab, Docker, Nexus, and Kubernetes. It empowers them to deliver high-quality software more rapidly.

\newpage
\section{Analysis and criticism of the current state}

While utilizing GitLab accelerates software development and deployment, there's a clear need for enhancements focused on sustainability and reducing environmental impact. An exhaustive study of the existing system was conducted to refine our project's scope and expected functionalities, aiming for a more reliable system than the previous one. Identified areas for improvement include

\begin{itemize}
  \item The current CI/CD approach relies heavily on resource-intensive technologies like Docker and Kubernetes, which contribute to increased energy consumption and carbon emissions.
  \item While effective for streamlining development workflows, the pipelines lack mechanisms for tracking and optimizing energy usage.
  \item Without visibility into the energy consumption and carbon footprint of the Kubernetes cluster, targeted strategies for reducing environmental impact are challenging to implement.
\end{itemize}

\subsection{Functional needs}
Functional requirements gathering is a crucial step in the project. This stage produces the functional specifications document, during which the expected functionalities are formalized along with all governing management rules. To address the issues identified in the existing study, the principle is to integrate monitoring tools such as Kepler (Kubernetes-based Efficient Power Level Exporter), Prometheus, and Grafana into the CI/CD pipeline. This integration aims to measure the energy consumption and CO$_2$ emissions of Orange's Kubernetes cluster, providing valuable insights for optimizing Green IT practices within the DevOps workflow.

\subsection{Non-Functional needs}

Non-functional requirements play a crucial role in system design,
because they define the attributes, constraints and restrictions to be taken into account.

These requirements, also called system qualities, guarantee the
user-friendliness and overall efficiency of the system. If any of these requirements are not met,
the system risks not meeting the internal needs of the company, users
or the market.

\begin{itemize}
  \item \textbf{Accuracy}: The system must provide accurate measurements of energy consumption and CO$_2$ emissions to support informed decision-making.
  \item \textbf{Scalability}: Ensure scalability to handle increasing data volumes as the size of the Kubernetes cluster grows.
  \item \textbf{Reliability}: The solution should be reliable, with minimal downtime, to maintain continuous monitoring and data collection.
  \item \textbf{Security}: Implement robust security measures to safeguard sensitive energy consumption and emission data from unauthorized access or tampering.
  \item \textbf{Performance}: Ensure efficient performance to deliver timely insights and visualizations, even during peak usage periods.
  \item \textbf{Ease of Use}: Design an intuitive user interface that allows administrators to easily configure monitoring settings and interpret data visualizations.
  \item \textbf{Compatibility}: Ensure compatibility with existing infrastructure and tools used within Orange Business Services Morocco, such as Prometheus, Grafana, and Kepler.
\end{itemize}


\section{DevOps appraoch}
The word DevOps is a combination of the terms development and operations, meant to represent a collaborative or shared approach to the tasks performed by a company's application development and IT operations teams.
The DevOps methodology aims to shorten the systems development lifecycle and provide continuous delivery with high software quality. These characteristics help ensure a culture of building, testing, and releasing software that is more reliable and at a high velocity.

\section{Definition of CI/CD}
\textit{Continuous Integration (CI)} is a software development practice where developers frequently merge their code changes into a central repository, typically multiple times a day. Each merge triggers an automated build and testing process. The primary goal of CI is to detect and address issues early in the development cycle, which helps in maintaining code quality and reducing integration problems. By integrating frequently, developers can identify bugs early, facilitating easier and quicker fixes.

\textit{Continuous Delivery (CD)} extends CI by automatically preparing code changes for a release to production. It ensures that the software can be reliably released at any time, with the deployment process being automated but requiring manual approval. Continuous Deployment is a further extension of CD where every change that passes all stages of the production pipeline is automatically released to customers without human intervention. This practice reduces the lead time for delivering new features and fixes, thus providing rapid feedback to developers and stakeholders.

\section{Definition of Green IT}

Green IT, also known as Green Information Technology, refers to environmentally sustainable computing practices. It encompasses a broad range of strategies aimed at minimizing the environmental impact of IT operations. This includes optimizing energy consumption, reducing electronic waste, and promoting the use of eco-friendly technologies. 

The goal of Green IT is to create more efficient and sustainable systems by implementing practices such as energy-efficient hardware, virtualization, and effective cooling systems in data centers. It also involves the adoption of cloud computing to optimize resource use and the implementation of robust recycling programs for electronic devices.

\section{Green DevOps:}
Green DevOps extends DevOps principles, focusing on resource efficiency while staying agile. It emphasizes sustainability, reducing environmental impact, and integrating eco-friendly strategies into software development. The goal is to balance technological innovation with ecological responsibility.
\begin{figure}[H]
  \centering
  \includegraphics[width=16cm]{Figures/devgreenops.png}
  \caption{Green DevOps}
\end{figure}
\subsection{The Fundamental Principles of Green DevOps}
Green DevOps is nothing more than an extension of DevOps. It shouldn't be viewed as an upheaval of all DevOps principles. We retain all DevOps principles but adjust them as finely as possible to optimize resource usage.

Green DevOps relies on the following concepts:
\begin{itemize}
  \item \textbf{Infrastructure Optimization:} Choosing eco-friendly data centers, energy-efficient hardware, and effective resource utilization are pillars of Green DevOps. The goal is to reduce energy consumption while maintaining performance.
  
  \item \textbf{Reduction of E-waste:} This involves minimizing over-provisioning of IT resources, properly managing electronic waste, and promoting hardware component recycling.
  
  \item \textbf{Automation and Process Optimization:} Automation is at the core of DevOps, but it holds even greater importance in the context of Green DevOps. Automating tasks reduces energy consumption by minimizing unused resources and enabling fine-grained resource management.
  
  \item \textbf{Carbon Impact Assessment:} To truly become "green," measurement is necessary. Green DevOps integrates tools and metrics to continuously assess and monitor the carbon footprint of projects. This helps identify areas needing improvement and track progress.
  
  \item \textbf{Reduction of Cycle Times:} Shorter development cycles mean less energy consumption and faster time to market. By reducing delays, Green DevOps contributes to the competitiveness of the business while minimizing its environmental impact.
  
  \item \textbf{Awareness:} DevOps teams need to be aware of environmental issues. Employee training and engagement are essential to fostering the adoption of sustainable practices.
\end{itemize}

\subsection{The Benefits of Green DevOps}

\begin{itemize}
  \item \textbf{Cost Reduction}: Less wasted resources, time saved, and efficient resource utilization result in a significant reduction in operating and infrastructure costs.
  
  \item \textbf{Quality Improvement}: Green DevOps practices encourage automation, monitoring, and rigorous testing, leading to better software quality, fewer bugs, and an enhanced user experience.
  
  \item \textbf{Social Responsibility}: Companies adopting Green DevOps demonstrate their commitment to environmental sustainability, which can enhance their brand image and attract new customers and talents.
  
  \item \textbf{Regulatory Compliance}: With an increasing number of environmental regulations in place, Green DevOps helps companies comply with standards and avoid potential penalties.
\end{itemize}

\section{Definition of DevOps Monitoring}
DevOps monitoring is the process of tracking and measuring the performance of applications and systems in order to help software development teams identify and resolve potential issues more quickly. This is typically done via a manual or automated DevOps monitoring solution or a collection of continuous monitoring tools that gather data

\subsection{DevOps Monitoring Use Cases}

The main benefit of DevOps monitoring is its ability to define, track, and measure KPIs across all aspects of DevOps. Here are some specific use cases of DevOps monitoring:

\begin{itemize}
  \item \textbf{Detect and Report Errors Earlier:}
  Flagging issues to DevOps teams more quickly means they can resolve them before they impact user experience. Early detection and reporting of errors allow for prompt resolution, minimizing disruptions and maintaining a smooth user experience.
  \item \textbf{Reduce System Downtime:}
  DevOps monitoring tools provide continuous oversight of databases, applications, and networks, enabling teams to resolve issues before system downtimes occur. By proactively identifying potential problems, teams can take preemptive actions to avoid outages and ensure system reliability.

  \item \textbf{Increase Security:}
  Through data analysis across the entire ecosystem, continuous monitoring in DevOps automates security measures by identifying inconsistencies or triggers that lead to security failures. Teams can respond to threats manually (on-call) or automatically with tools, enhancing the overall security posture and protecting sensitive data.

  \item \textbf{Enhance Observability of DevOps Components:}
  Easily identify when various systems and applications in your DevOps stack degrade in performance, cost, security, or other factors to avoid problems down the road. Enhanced observability enables teams to maintain optimal performance and security by monitoring and analyzing system behavior continuously.

  \item \textbf{Uncover Root Cause of Issues Faster:}
  Continuous tracking of logs and metrics helps teams identify the root cause — where a problem started or occurred. This allows engineers to detect patterns in system behavior to anticipate and prevent future issues, improving mean time to detection (MTTD), mean time to repair (MTTR), and mean time to isolate (MTTI).
\end{itemize}

\section{}

\subsection{Subsection1}
\begin{longtable}[c]{| m{4.4cm} | m{11cm} |}
\caption{Long table 1}\\
 \hline

 Cell & Description  \\ 
 \hline
 \endfirsthead

 \hline
 
 Cell & Description  \\ 
 \hline
 \endhead

        \hline
          Element11 & Element21 \\
        \hline
          Element12 & Element22 \\
        \hline
          Element13 & Element23 \\
        \hline
          Element14 & Element24 \\
        \hline
          Element15 & Element25 \\
        \hline
          Element16 & Element26 \\
        \hline
          Element17 & Element27 \\
        \hline
          Element18 & Element28 \\
        \hline
          Element19 & Element29 \\
        \hline
          Element110 & Element210 \\
        \hline
          Element111 & Element211 \\
        \hline
          Element112 & Element212 \\
        \hline
          Element113 & Element213 \\
        \hline
          Element114 & Element214 \\
        \hline

 \end{longtable}

 
\subsection{Subsection2}

\subsubsection{Subsubsection1}

\subsubsection{Subsubsection2}

\paragraph{Paragraph a}

\paragraph{Paragraph b}

\section{Section2}

\subsection{Subsection1}

\subsection{Subsection2}

\subsubsection{Subsubsection1}

\subsubsection{Subsubsection2}

\paragraph{Paragraph a}

\paragraph{Paragraph b}


\newpage

\section*{Conclusion}


Lorem ipsum dolor sit amet, consectetur adipiscing elit. Praesent nec dapibus justo. Donec sagittis vulputate ante sed porttitor. Suspendisse sit amet nisl massa. Curabitur nec nisl condimentum, egestas ex vitae, dapibus enim. Etiam iaculis, erat faucibus pellentesque sagittis, nisi justo sollicitudin nibh, et condimentum augue massa non turpis. Proin commodo enim fermentum suscipit condimentum. Maecenas molestie, dui nec vestibulum rhoncus, arcu nisl faucibus neque, a ornare nisi massa ac eros. Aenean id velit sit amet lacus mattis varius. Donec fringilla massa sed nisi eleifend, a aliquet mi tempus. Nunc posuere euismod est, nec tristique augue lobortis non. Sed sodales sem ut metus tempus ullamcorper.




























\chapter{Technologies Used}
\label{chap:Chapter 3 title}
The adoption of the DevOps approach largely relies on the automation of communication between the various required tools, thus allowing us to benefit from the previously described functionalities. The choice of tools for each stage of our DevOps pipeline, such as continuous integration and continuous deployment, depends on specific criteria related to our needs. Consequently, this comparative study aims to reinforce the selection of tools that we have implemented for the integration of Green IT into DevOps practices.

\newpage


\section{Application Containerization}
In this section, we will delve into our problem by examining the fundamental principles of traditional virtualization and containerization. Our aim is to provide a clear description of these concepts and explain the relationships between them.

\subsection{Virtualization}
Virtualization involves creating a virtual version of a device or resource, such as an operating system, server, storage device, or network resource. It allows for the abstraction of physical resources, representing them independently of their physical equivalents. Each virtual element, such as a disk, network interface, local network, switch, processor, or memory, is associated with a physical resource in a computer system. Thus, virtual machines hosted by the host machine are considered applications that require allocation or distribution of the host's resources. There are various types of virtualization, each corresponding to a specific use case:

\begin{itemize}
    \item \textbf{System Virtualization} is a server virtualization technique that uses a hypervisor to allow the host machine to run multiple virtual instances simultaneously. These virtual instances are commonly referred to as virtual machines (VMs), while the hypervisor is known as the virtual machine monitor (VMM), responsible for managing the VMs.
    \item \textbf{Application Virtualization} differs from system virtualization by using a virtualization layer in the form of an application. This application creates virtual instances and isolates them from the specifics of the host machine, such as its architecture or operating system. This approach allows developers to avoid creating multiple versions of their software to run in different environments. Common application virtual machines include the JVM (Java Virtual Machine) and containers.
\end{itemize}

\subsection{Containerization}
Container-based virtualization, also known as containerization, offers an alternative to system virtualization. This approach directly leverages kernel features to create isolated virtual environments called containers. Containers use control groups (cgroups) and namespaces provided by the operating system kernel. Namespaces allow for controlling and limiting the resources used by a process, while cgroups manage the resources of a group of processes. Thus, a container provides the necessary resources to run applications in isolation, as if they were the only processes running on the host machine's operating system.

Containers offer advantages such as scalable, modular, and loosely coupled application development. The need to create portable deliverables independent of development or production environments has become a standard. The term "container" emerged to refer to an application independent of the system, represented as a box containing all the dependencies necessary for its proper functioning.

\subsection{Containerization vs Virtualization}
Containers have an intrinsically smaller footprint than virtual machines and require less startup time. This means that more containers can run on the same computing capacity compared to a single virtual machine. This improvement in server efficiency reduces costs associated with servers and licenses. In simple terms, containerization allows applications to be developed once and run anywhere. This portability is crucial for streamlining the development process and ensuring compatibility with different service providers.

Figure \ref{fig:container_vs_vm} illustrates the distinction between container architecture and virtual machine architecture.

\begin{figure}[h]
    \centering
    \includegraphics[width=16cm]{Figures/containers-vs-vms.jpg}
    \caption{Comparison between Container and Virtual Machine Architectures}
    \label{fig:container_vs_vm}
\end{figure}

\subsection{Containerization Technologies}
There are several containerization tools available in the market. We conducted a comparison to select the most suitable one.

\begin{table}[h]
    \centering
    \begin{tabular}{|p{4cm}|p{4cm}|p{4cm}|p{4cm}|}
        \hline
        \textbf{Technology} & \textbf{Principle} & \textbf{Market Position} & \textbf{Documentation/ Complexity} \\
        \hline
        \textbf{Docker} & A container contains a single process; multiple containers can be connected to form an application. & Market leader, widely used by major online service providers (e.g., Netflix, Spotify) & Official documentation is available; community-driven tutorials and resources are abundant. Known for its simplicity.\\
        \hline
        \textbf{Podman} & A container contains a single process; multiple containers can be connected to form an application. & Increasingly popular, supported by major enterprises like Google and Red Hat. Acquired by Red Hat in May 2018. & Limited documentation, somewhat difficult due to lack of resources but highly modular and compatible with ancillary tools.\\
        \hline
        \textbf{LXC/LXD} & Isolation of an entire application within a complete Linux environment (similar to a VM). & Established since 2008, the only one closely resembling a VM. Isolated but with an active community. & Limited documentation; most information is posted on Ubuntu community forums. \\
        \hline
    \end{tabular}
    \caption{Comparison of Containerization Tools}
    \label{tab:container_tools_comparison}
\end{table}

The analysis presented in Table \ref{tab:container_tools_comparison} highlights the significance of Docker in the market. Docker stands out as the most universal solution in terms of image compatibility and offers numerous additional advantages. It simplifies the creation and management of containers, facilitates the design and construction of images, and enables easy distribution and version control of these images. Thus, Docker plays a crucial role in streamlining and enhancing containerization processes.

\section{Introduction to Docker}
Docker uses a client-server architecture. The Docker client talks to the Docker daemon, which does the heavy lifting of building, running, and distributing your Docker containers. The Docker client and daemon can run on the same system, or you can connect a Docker client to a remote Docker daemon. The Docker client and daemon communicate using a REST API, over UNIX sockets or a network interface. Another Docker client is Docker Compose, that lets you work with applications consisting of a set of containers.
\newpage
\begin{itemize}
  \item \textbf{Docker Daemon (\textit{dockerd}):} Responsible for handling Docker API requests and managing various Docker objects such as images, containers, networks, and volumes. Listens for incoming requests and executes them accordingly. Can interact with other daemons to manage Docker services across different environments.
  
  \item \textbf{Docker Client (\textit{docker}):} Primary interface for Docker users to interact with the Docker ecosystem. Users issue commands through the Docker client (e.g., \texttt{docker run}), which are communicated to the daemon for execution. Utilizes the Docker API for communication and can connect with multiple daemons for managing Docker instances.
  
  \item \textbf{Docker Registries:} Facilitate storage and distribution of containerized applications. Docker Hub is a public registry accessible to all users, while private registries can be set up for hosting proprietary or sensitive images. Commands like \texttt{docker pull} or \texttt{docker run} fetch required images from the configured registry, and \texttt{docker push} uploads images to the designated registry for sharing or backup purposes.
\end{itemize}

\begin{figure}[h]
  \centering
  \includegraphics[width=\textwidth]{Figures/docker-architecture.png}
  \caption{Docker Architecture}
  \label{fig:container_architecture}
\end{figure}

\section{Container Orchestration}

Container orchestration involves automating much of the operational tasks required to run containerized workloads and services. This encompasses various aspects that software teams need to manage the lifecycle of a container, such as provisioning, deployment, scaling (both up and down), network management, and load balancing. Container orchestration simplifies and optimizes container management by providing tools and features to effectively coordinate different container instances and associated resources. This facilitates the deployment and management of complex and scalable containerized applications in production environments.

\subsection{Why Orchestration is Necessary}

The ability to create containers has existed for several decades but became widely accessible in 2008 when Linux integrated container functionality into its kernel. However, widespread adoption of containers took off with the introduction of Docker, an open-source containerization platform, in 2013. Docker became so popular that the terms "Docker containers" and "containers" are often used interchangeably.

Containers have become the de facto computing units for modern cloud-native applications due to their small size, resource efficiency, and portability compared to virtual machines (VMs). Containerized microservices or serverless functions are particularly prevalent in these applications. Manually deploying and managing a small number of containers is relatively easy. However, in most organizations, the number of containerized applications is rapidly increasing, making large-scale management impossible without automation. This is where container orchestration comes into play.

Container orchestration provides essential automation for managing a large number of containerized applications. This significantly reduces the effort and complexity associated with managing this application fleet. By automating operations, orchestration facilitates an agile or DevOps approach, allowing development teams to rapidly deploy new features and capabilities.

The automation provided by orchestration enhances many inherent benefits of containerization. For example, it enables automated host selection and efficient resource allocation based on declarative configuration, thereby maximizing the utilization of computing resources. Additionally, orchestration provides automated monitoring of container health status and can move containers as needed to maximize availability.

In summary, container orchestration brings essential automation to effectively manage a large number of containerized applications. It facilitates rapid and iterative development cycles, allowing teams to release new features and capabilities more quickly. Furthermore, orchestration can enhance and extend the benefits of containerization through advanced features such as resource management and automated monitoring.

\subsection{Kubernetes Container Orchestration}

Kubernetes is a powerful and popular container orchestration platform that enables enterprises to provide a highly productive PaaS for developing cloud-native applications. With Kubernetes, development teams can focus on coding and innovation, while container infrastructure and operations are managed automatically and efficiently. Kubernetes' advantages over other orchestration solutions (such as Docker Swarm, Apache Mesos, OpenShift, AWS Fargate) largely stem from its various more comprehensive and sophisticated features in several areas, including:

\begin{itemize}
    \item \textbf{Container Deployment:} Kubernetes deploys a specified number of containers on a given host and maintains them in the desired state.
    \item \textbf{Rollouts:} A rollout is a modification to a deployment. Kubernetes allows launching, pausing, resuming, or canceling deployments.
    \item \textbf{Service Discovery:} Kubernetes can automatically expose a container to the internet or other containers using a DNS name or IP address.
    \item \textbf{Storage Provisioning:} Developers can configure Kubernetes to mount local or cloud-based persistent storage for containers as needed.
    \item \textbf{Load Balancing and Scaling:} When traffic to a container experiences a high spike, Kubernetes can employ load balancing and scaling to distribute it across the network to ensure stability and performance. (This also saves developers from having to configure a load balancer)
    \item \textbf{Self-healing for High Availability:} When a container fails, Kubernetes can automatically restart or replace it. It can also take containers out of service that do not meet our health check requirements.
    \item \textbf{Multi-cloud Provider Support and Portability:} Kubernetes enjoys broad support from all major cloud providers. This is particularly important for organizations deploying applications in a hybrid or multi-cloud environment.
\end{itemize}

\section{Helm}

\begin{figure}[H]
  \centering
  \includegraphics[width=5cm]{Figures/helm-logo.png}
  \caption{Helm}
  \label{fig:helm}
\end{figure}

Helm is a package manager for Kubernetes, facilitating the deployment, management, and scaling of applications within a Kubernetes cluster. Helm simplifies complex Kubernetes application deployments by providing a higher level of abstraction through Helm charts. These charts are pre-configured packages of Kubernetes resources.

\subsection{Key Features of Helm}
\begin{itemize}
    \item \textbf{Charts:} Helm uses charts to define, install, and upgrade even the most complex Kubernetes applications. A Helm chart is a collection of files that describe a related set of Kubernetes resources.
    \item \textbf{Releases:} When a chart is deployed, it becomes a release. Helm manages these releases, enabling rollbacks and upgrades.
    \item \textbf{Repositories:} Helm charts can be stored and shared via repositories, making it easy to distribute applications.
    \item \textbf{Templates:} Helm charts support templates, which allow for dynamic and customizable Kubernetes manifests. Templates are written in Go templating syntax.
    \item \textbf{Lifecycle Management:} Helm provides commands to manage the entire lifecycle of applications, from installation and upgrades to rollbacks and deletion.
\end{itemize}

\subsection{Benefits of Using Helm}
\begin{itemize}
    \item \textbf{Simplified Deployments:} Helm abstracts the complexity of Kubernetes configurations, making deployments more straightforward and repeatable.
    \item \textbf{Version Control:} Helm charts allow you to manage versions of your application, enabling easy rollbacks and upgrades.
    \item \textbf{Sharing and Reusability:} Helm charts can be shared across teams and organizations, promoting reusability and standardization.
    \item \textbf{Scalability:} Helm facilitates the management of applications at scale, handling multiple Kubernetes resources efficiently.
\end{itemize}

Helm significantly enhances the Kubernetes experience by providing a robust framework for managing the complexities of application deployment and lifecycle management. It empowers developers to create more consistent and reliable deployments while reducing the potential for human error.


\section{Source Code Management Tools}

Source Code Management (SCM) concerns how software changes are managed. Its primary goal is to accelerate the delivery of quality code changes by development teams. SCM tools enhance tracking, visibility, collaboration, and control throughout the software development lifecycle, allowing developers working on complex projects to be more creative, have more freedom, and options.

By using SCM, source files are protected against anomalies, and all teams can identify who made which changes and at what stage of the process. This promotes transparency and facilitates collaboration among team members.

We conducted a source code manager study to consolidate the choice of the GitLab tool, which is already used by Orange Business, compared to its competitors BitBucket and GitHub.

Table 3.2 summarizes the comparison of the previously presented source code managers:

\begin{table}[h]
  \centering
  \resizebox{\textwidth}{!}{%
    \begin{tabular}{|c|c|c|c|c|}
      \hline
      Tool & Open Source & License & Version Control & CD integration \\
      \hline
      \includegraphics[height=1cm]{Logos/gitlab-logo.png} & Yes & Free for community version & Git & Already integrated or through other tools \\
      \hline
      \includegraphics[height=1cm]{Logos/github-logo.png} & No & Paid & Git & Integration through other tools or GitHub Actions \\
      \hline
      \includegraphics[height=1cm]{Logos/bitbucket-logo.png} & Yes & Free & Git & Integration through other tools or Jira \\
      \hline
    \end{tabular}
  }
  \caption{Comparison of source code management tools}
\end{table}

We chose GitLab without hesitation due to its open-source nature and free community version. This decision was motivated by the ability to integrate continuous delivery and issue tracking functionalities, which will allow us to expand our tool usage in the future.

\subsection{GitLab Platform}

GitLab is now a comprehensive DevOps platform presented as a single application. This platform revolutionizes development, security, operations, and collaboration practices within teams. With GitLab, it is possible to create, test, and deploy software faster, using an integrated solution. GitLab includes the following features:
\begin{itemize}
    \item \textbf{Source Code Management:} Facilitates version control, collaboration, and accelerates the delivery of high-performance software. Enables efficient task coordination, tracking, and verification of changes, as well as simplified code reviews.
    \item \textbf{Agile Project Management:} Improves visibility within the organization, helping to meet set deadlines and budgets. Promotes collaboration among different teams.
    \item \textbf{Continuous Integration and Continuous Delivery (CI/CD):} GitLab's CI/CD feature enables the creation of high-quality applications by automating integration and deployment processes.
    \item \textbf{Cost-Efficient Software Deployment:} GitLab enables faster software releases with reduced development costs.
\end{itemize}

\section{Prometheus}

\begin{figure}[H]
  \centering
  \includegraphics[width=4cm]{Logos/prometheus-logo.png}
  \caption{Prometheus}
  \label{fig:prometheus_logo}
\end{figure}

Prometheus is an open-source monitoring and alerting toolkit designed for reliability and scalability in modern, dynamic environments. It is part of the Cloud Native Computing Foundation (CNCF) and is widely used for monitoring containerized applications and microservices architectures. Prometheus provides a multi-dimensional data model with time series data identified by metric name and key/value pairs. It collects metrics from configured targets at specified intervals, evaluates rule expressions, displays the results, and can trigger alerts if specified conditions are met.

\subsection{Prometheus Architecture}

\begin{figure}[H]
  \centering
  \includegraphics[width=16cm]{Figures/prometheus-grafana-architecture-diagram.png}
  \caption{Prometheus Architecture}
  \label{fig:prometheus_architecture}
\end{figure}

Figure \ref{fig:prometheus_architecture} illustrates the Prometheus architecture. Prometheus discovers targets to scrape via service discovery, which can include instrumented applications or third-party applications accessed through exporters. Scraped data is stored and can be utilized for dashboards using PromQL or for sending alerts to the Alertmanager, which converts them into various notifications.

\begin{itemize}
  \item \textbf{Client Libraries:} Client libraries allow for easy instrumentation of applications, typically requiring only a few lines of code. Prometheus provides official client libraries in Go, Python, Java/JVM, Ruby, and Rust, with third-party libraries available for other languages. These libraries handle thread safety, bookkeeping, and producing the Prometheus text format. They can automatically pick up metrics from dependencies and offer built-in metrics for CPU usage and garbage collection.
  
  \item \textbf{Exporters:} Exporters collect metrics from software that cannot be instrumented directly. They transform data from an application's metrics interface into the Prometheus exposition format. Exporters are widely available and can be easily extended if necessary.
  
  \item \textbf{Service Discovery:} Prometheus uses service discovery to locate and monitor applications and exporters. Integrations are available for Kubernetes, EC2, Consul, and more. Service discovery information is mapped to monitoring targets and their labels using relabeling.
  
  \item \textbf{Scraping:} Prometheus fetches metrics by sending HTTP requests called scrapes. The responses are parsed and stored. Scraping is configured to occur regularly, typically every 10 to 60 seconds per target.
  
  \item \textbf{Storage:} Prometheus stores data locally in a custom database, optimized for high performance. The storage system in Prometheus 2.0 can ingest millions of samples per second with efficient compression.
  
  \item \textbf{Dashboards:} Prometheus provides HTTP APIs for querying and displaying data. Grafana is recommended for creating advanced dashboards, supporting multiple Prometheus servers.
  
  \item \textbf{Recording Rules and Alerts:} Recording rules allow regular evaluation of PromQL expressions, storing the results. Alerting rules work similarly, generating alerts sent to the Alertmanager.
  
  \item \textbf{Alert Management:} The Alertmanager handles alerts from Prometheus servers, converting them into notifications via email, chat applications, and services like PagerDuty. Alerts can be aggregated, throttled, and routed based on different team requirements.
  
  \item \textbf{Long-Term Storage:} Prometheus stores data locally, limiting retention to available disk space. For long-term storage, remote read and write APIs enable integration with external systems for extended data retention.
\end{itemize}



\section{Grafana}

  
\begin{figure}[H]
  \centering
  \includegraphics[width=5cm]{Logos/grafana-logo.png}
  \caption{Grafana}
\end{figure}
Grafana is an open-source analytics and visualization platform commonly used alongside Prometheus for monitoring and observability purposes.
Grafana provides a rich set of features for creating interactive dashboards and visualizations from various data sources, including Prometheus metrics, logs, and other time series databases.

\begin{itemize}
  \item \textbf{Dashboards:} With Grafana, users can easily create and customize dashboards to visualize key performance indicators, monitor system health, and track application metrics in real-time.
  \item \textbf{Visualization Options:} Grafana supports a wide range of visualization options, including graphs, tables, heatmaps, and histograms, allowing for flexible and insightful data analysis.
  \item \textbf{Integration:} Grafana offers extensive integration capabilities with numerous data sources, including Prometheus, Elasticsearch, Graphite, and InfluxDB, making it a versatile tool for consolidating and visualizing data from different sources in a single interface.
  \item \textbf{Advanced Features:} Grafana provides advanced features such as alerting and annotations, allowing users to set up alerts based on predefined thresholds or conditions and annotate dashboards with contextual information for improved situational awareness.
\end{itemize}

\section{Project Management Tool}

With the rapid evolution of information and communication technologies, user needs are increasing and becoming more demanding, while the economic context is constantly changing. In this context, managing IT projects has become a challenge for companies, as it is essential to master and successfully complete them, regardless of their size or type. For our DevSecOps team, which adopts the Agile Scrum method, Jira Software offers ready-to-use Scrum and Kanban boards. These boards serve as task management centers, where tasks are associated with customizable workflows. They ensure transparency of teamwork and provide visibility into the progress of each task. Time tracking features and real-time performance reports (Burnup/Burndown charts, sprint reports, velocity charts) allow the team to closely monitor its productivity over time.

\begin{figure}[h]
    \centering
    \includegraphics[width=0.5\textwidth]{Logos/jira-png.png}
    \caption{Jira logo}
\end{figure}

\section{Conclusion}

Throughout this chapter, we have synthesized the selection of various tools necessary for the implementation of our project. By comparing the tools that correspond to our needs, we can detail the phases of implementing these solutions in the next chapter with the goal of achieving our objectives.

\pagebreak

\chapter{Implementation}
\label{chap:Chapter 4 title}
\section*{Introduction}

The implementation chapter is a crucial step in our project. In this chapter, we will present in detail the various steps we followed to implement our solution. Overall, this implementation chapter will allow us to document in a detailed and transparent manner the process of executing our project. It will highlight the efforts made, the choices made, and the results obtained.

\pagebreak

\section{Creating a GitLab Template for Kepler}

To successfully integrate Kepler with our GitLab CI pipeline and ensure a seamless, efficient, and automated workflow, we followed a series of steps that included configuring the necessary environments, creating precise configuration files, and integrating essential monitoring tools
\subsection*{Configuration of the Environment}

The initial step was to set up the execution environment for GitLab CI. We configured the runners, which are the virtual machines or containers responsible for executing the CI jobs. It was crucial to ensure that all dependencies required to run Kepler, Prometheus, and Grafana were installed on these runners.

\subsection*{Creation of the CI Configuration File}

We added a \texttt{template-ci-kepler.yml} file. This file plays a 
role as it defines the various stages of the CI pipeline and specifies the actions required to integrate Kepler into our CI process. By using this configuration file, we were able to clearly and systematically outline the tasks necessary for seamless integration into our pipeline.

 \begin{figure}[H]
    \centering
    \includegraphics[width=19cm]{Figures/kepler-template-gitlab.png}
    \caption{Kepler template repository}
\end{figure}

\subsection{Integration of the Kepler Template}

After creating the template, it can be used and integrated directly into a project. To achieve this, a few lines must be added to the \texttt{.gitlab-ci.yml} file.

First, we add the \texttt{include} directive followed by the path to our directory containing the Kepler template into an existing pipeline.

\begin{figure}[H]
  \centering
  \includegraphics[width=13cm]{Figures/kepler-template-integration.png}
  \caption{Integration of the Kepler template}
\end{figure}

Next, we specify the stage name as \texttt{"deploy"}.

\begin{figure}[H]
  \centering
  \includegraphics[width=10cm]{Figures/deploy-kepler-stage.png}
  \caption{Stage \texttt{"deploy"}}
\end{figure}

We define the following variables for the deployment:

\begin{itemize}
    \item \texttt{HELM\_RELEASE\_NAME} as the name of our Helm release.
    \item \texttt{HELM\_NAMESPACE} as the namespace for our Helm deployment.
    \item \texttt{KUBECONFIG\_CONTENT} to store cluster authentication information for kubectl.
\end{itemize}

\begin{figure}[H]
  \centering
  \includegraphics[width=0.8\textwidth]{Figures/deployment-variable.png}
  \caption{Deployment variables for Kepler}
\end{figure}

This configuration allows us to include the steps for deploying Kepler in our CI/CD pipeline using Helm and Kubernetes.


\section{Deploying Prometheus \& Grafana on Kubernetes}

\section{Configuring Rules and Alerts in Prometheus}
\subsection{Setting Up Alerting Rules}
\subsection{Configuring Alerts in Prometheus}

\section{Slack Integration with Prometheus Alerts}
\subsection{Configuring Slack for Prometheus Alerts}
\subsection{Testing Slack Integration}

\section{Conclusion}







\newpage

\section*{Conclusion}

Lorem ipsum dolor sit amet, consectetur adipiscing elit. Praesent nec dapibus justo. Donec sagittis vulputate ante sed porttitor. Suspendisse sit amet nisl massa. Curabitur nec nisl condimentum, egestas ex vitae, dapibus enim. Etiam iaculis, erat faucibus pellentesque sagittis, nisi justo sollicitudin nibh, et condimentum augue massa non turpis. Proin commodo enim fermentum suscipit condimentum. Maecenas molestie, dui nec vestibulum rhoncus, arcu nisl faucibus neque, a ornare nisi massa ac eros. Aenean id velit sit amet lacus mattis varius. Donec fringilla massa sed nisi eleifend, a aliquet mi tempus. Nunc posuere euismod est, nec tristique augue lobortis non. Sed sodales sem ut metus tempus ullamcorper.

\pagebreak

\input{Chapters/Chapter5}


\chapter*{General Conclusion and Perspectives}


% This conclusion is unnumbered, if you want it numbered, you can remove the * from above and remove the line below, so it becomes a chapter, then add sections.

\addcontentsline{toc}{chapter}{General Conclusion and Perspectives} %adds to the table of contents 

\label{chap:General Conclusion} 

Lorem ipsum dolor sit amet, consectetur adipiscing elit. Praesent nec dapibus justo. Donec sagittis vulputate ante sed porttitor. Suspendisse sit amet nisl massa. Curabitur nec nisl condimentum, egestas ex vitae, dapibus enim. Etiam iaculis, erat faucibus pellentesque sagittis, nisi justo sollicitudin nibh, et condimentum augue massa non turpis. Proin commodo enim fermentum suscipit condimentum. Maecenas molestie, dui nec vestibulum rhoncus, arcu nisl faucibus neque, a ornare nisi massa ac eros. Aenean id velit sit amet lacus mattis varius. Donec fringilla massa sed nisi eleifend, a aliquet mi tempus. Nunc posuere euismod est, nec tristique augue lobortis non. Sed sodales sem ut metus tempus ullamcorper.

Nullam fermentum id mauris suscipit varius. Quisque tristique tempor fringilla. Nam porta tincidunt orci in consectetur. Suspendisse nec sem nisi. Suspendisse potenti. Sed sodales aliquam libero, at dapibus ligula accumsan non. Quisque congue et urna a consectetur. Nullam maximus venenatis ornare. Donec in luctus urna, vel rhoncus lectus. Etiam lobortis, lacus nec gravida iaculis, magna nulla iaculis leo, quis congue nunc tortor pellentesque lorem. Sed fermentum vulputate dui, ac malesuada eros molestie sit amet. Proin fringilla elit justo, in posuere urna porttitor et. Curabitur dictum justo vitae metus pellentesque, eget feugiat sem feugiat.

Curabitur in mauris eu felis cursus auctor eget ut massa. Curabitur eleifend consectetur magna in ultrices. Etiam ut semper turpis. Morbi sed ipsum nec sem rutrum blandit sit amet vitae felis. Vestibulum vitae hendrerit diam. Aliquam pellentesque est mi, et tempus nisl laoreet in. Maecenas consequat augue a ante congue dignissim. In condimentum erat in volutpat tempus. Mauris vulputate, enim ut dignissim varius, lorem purus tristique sapien, lobortis accumsan dolor lorem quis eros. Duis vel imperdiet metus, non suscipit tortor.


%\section{Optional Section}

\appendix


\chapter{Glossary}
\label{chap:Glossary} 


\textbf{Telnet:} Telnet is a network protocol that allows users to establish a remote terminal connection to a computer or network device over a network.\\


\textbf{SSH (Secure Shell):} SSH is a network protocol used for secure remote administration and secure file transfers.\\


\textbf{SNMP (Simple Network Management Protocol):} SNMP is a network protocol designed for managing and monitoring network devices and systems.\\


\pagebreak

\chapter{Acronyms}

%Example of Acronyms using table 

\begin{tabular}{l l}

% page 1

\textbf{IP} & Internet Protocol \\
\textbf{CNS} & Cloud and Network Services \\
\textbf{MN} & Mobile Networks \\
\textbf{NI} & Network Infrastructure \\
\textbf{IoT} & Internet of Things \\
\textbf{NFV} & Network Functions Virtualization \\
\textbf{SDN} & Software-Defined Network \\
\textbf{TCP} & Transmission Control Protocol \\
\textbf{7750 SR} & Nokia 7750 Service Router \\
\textbf{OSPF} & Open short Path First \\
\textbf{BGP} & Border Gateway Protocol \\
\textbf{VPN} & Virtual Private Network \\
\textbf{MPLS} & Multiprotocol Label Switching \\
\textbf{EVE-NG} & Emulated Virtual Environment - Next Generation \\
\textbf{VM} & Virtual Machine \\
\textbf{SSH} & Secure Shell \\
\textbf{WinSCP} & Windows Secure Copy \\
\textbf{FTP} & File Transfer Protocol \\
\textbf{SSL} & Secure Sockets Layer \\
\textbf{TLS} & Transport Layer Security \\
\textbf{FTPS} & FTP over SSL/TLS \\


% end of page 1

\end{tabular}

\pagebreak

\begin{tabular}{l l}

%page 2


\textbf{SFTP} & Secure File Transfer Protocol \\
\textbf{SCP} & Secure Copy Protocol \\
\textbf{CLI} & Command-line interface \\
\textbf{MD-CLI} & Model-Driven Command Line \\
\textbf{API} & Application Programming Interface \\
\textbf{re} & Regular Expression \\
\textbf{IGP} & Internal Gateway Protocol \\
\textbf{IS-IS} & Intermediate-System Intermediate-System \\
\textbf{SPF} & Shortest Path First \\
\textbf{SF} & Switch Fabric \\
\textbf{CPM} & Control Processing Module \\
\textbf{IOM} & Input/Output Modules \\
\textbf{MDA} & Media Dependent Adapters \\
\textbf{QoS} & Quality of Service \\
\textbf{IPsec} & Internet Protocol Security \\
\textbf{VSR} & Virtual Services Router \\
\textbf{DDoS} & Distributed Denial-of-Service \\
\textbf{SROS} & Service Router Operating System \\
\textbf{NETCONF} & Network Configuration Protocol \\
\textbf{YANG} & Yet Another Next Generation \\
\textbf{MP-BGP} & Multiprotocol Border Gateway Protocol \\
\textbf{IPv6} & Internet Protocol Version 6 \\
\textbf{AS} & Autonomous Systems \\
\textbf{iBGP} & Internal Border Gateway Protocol \\
\textbf{eBGP} & External Border Gateway Protocol \\
\textbf{LAG} & Link Aggregation Group \\
\textbf{VPRN} & Virtual Private Routed Network \\
\textbf{SNMP} & Simple Network Management Protocol \\
\textbf{Telnet} & Telecommunication Network \\
\textbf{YAML} & Yet Another Markup Language \\
\textbf{XML} & eXtensible Markup Language \\

%end of page 2

\end{tabular}

\chapter{Some subject you want to expand on}
\label{chap:Some subject you want to expand on} 


You can add more appendices depending on the subjects.
%You can decide which code snippets to use in the main report and which to add here in the annex
    
    \lstset{style=mystyle} %this style is already defined in Packages.tex
    
    \begin{lstlisting}[language=bash, caption= Bash example]
    
    [root@host ~]# You can change the language and caption.

    \end{lstlisting}



    \begin{lstlisting}[language=python, caption= Python example]
    
    # Solve the quadratic equation ax**2 + bx + c = 0

    # import complex math module
    import cmath
    
    a = 1
    b = 5
    c = 6
    
    # calculate the discriminant
    d = (b**2) - (4*a*c)
    
    # find two solutions
    sol1 = (-b-cmath.sqrt(d))/(2*a)
    sol2 = (-b+cmath.sqrt(d))/(2*a)
    
    print('The solution are {0} and {1}'.format(sol1,sol2))

    \end{lstlisting}



\begin{table}[H]
    \centering
    \begin{tabular}{|m{5cm}|m{10cm}|}
        \hline
          Column1 & Column2 \\
        \hline
          Element11 & Element21 \\
        \hline
          Element12 & Element22 \\
        \hline
          Element13 & Element23 \\
        \hline
    \end{tabular}
    \caption{Table Example}
\end{table}




\begin{longtable}[c]{| m{4.4cm} | m{11cm} |}
\caption{Long table Example}\\
 \hline

 Cell & Description  \\ 
 \hline
 \endfirsthead

 \hline
 
 Cell & Description  \\ 
 \hline
 \endhead

        \hline
          Element11 & Element21 \\
        \hline
          Element12 & Element22 \\
        \hline
          Element13 & Element23 \\
        \hline
          Element14 & Element24 \\
        \hline
          Element15 & Element25 \\
        \hline
          Element16 & Element26 \\
        \hline
          Element17 & Element27 \\
        \hline
          Element18 & Element28 \\
        \hline
          Element19 & Element29 \\
        \hline
          Element110 & Element210 \\
        \hline
          Element111 & Element211 \\
        \hline
          Element112 & Element212 \\
        \hline
          Element113 & Element213 \\
        \hline
          Element114 & Element214 \\
        \hline

 \end{longtable}


\begin{thebibliography}{99}
\addcontentsline{toc}{chapter}{Bibliography}


\bibitem{ref1}
Author name, Book name.

\bibitem{ref2}
\emph{Title 1},
\href{https://www.overleaf.com/learn/latex/Bibliography_management_with_bibtex}{\textbf{Title 2}}

 %the link is a documentation of the basic bibliography method (that    I'm using here) + bibTex which is more advanced, read it well and decide which one works best for you.



\end{thebibliography}

\end{document}