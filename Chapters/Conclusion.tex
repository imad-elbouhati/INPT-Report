\chapter*{General Conclusion and Perspectives}


% This conclusion is unnumbered, if you want it numbered, you can remove the * from above and remove the line below, so it becomes a chapter, then add sections.

\addcontentsline{toc}{chapter}{General Conclusion and Perspectives} %adds to the table of contents 

\label{chap:General Conclusion} 

In conclusion, the project focused on collecting and estimating power consumption metrics in cloud environments using Kepler to monitor and optimize energy usage. Kepler proved effective in bare-metal environments by directly collecting real-time power metrics and utilizing the Ratio Power model for accurate process-level power consumption estimation. In virtual machine (VM) environments, where direct measurement is challenging, Kepler employs trained power models to estimate energy consumption, ensuring continuous and reliable monitoring.

Integrating Kepler into our development workflow has enhanced our understanding and management of energy consumption, leading to resource optimization and reduced energy costs. This approach also promotes a more sustainable and environmentally friendly management of cloud infrastructures.

Looking ahead, we plan to further explore the advanced features of Kepler, particularly improving power models for VM environments and integrating new technologies for better accuracy and efficiency. We also aim to expand our CI/CD pipeline to include additional monitoring and optimization tools, thereby enhancing our ability to detect and address energy inefficiencies proactively.

Additionally, we seek to strengthen collaboration between development, infrastructure and Green IT teams to foster an integrated and proactive approach to energy consumption management throughout the application lifecycle. This synergy between teams will contribute to the continuous improvement of our practices and the achievement of our sustainability and energy efficiency goals.

%\section{Optional Section}